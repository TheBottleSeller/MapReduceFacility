\documentclass[11pt]{article}

\usepackage{amsmath}
\usepackage{amssymb}
\usepackage{fancyhdr}
\usepackage{enumerate}

\oddsidemargin0cm
\topmargin-2cm 
\textwidth16.5cm
\textheight23.5cm  

\newcommand{\question}[2] {\vspace{.25in} \hrule\vspace{0.5em}
\noindent{\bf #1: #2} \vspace{0.5em}
\hrule \vspace{.10in}}
\renewcommand{\part}[1] {\vspace{.10in} {\bf (#1)}}

\newcommand{\qed}{\hfill \ensuremath{\Box}}

\setlength{\parindent}{0pt}
\setlength{\parskip}{5pt plus 1pt}

\begin{document}

\medskip

\thispagestyle{plain}
\begin{center}
{\Large 15-440 Lab 3: MapReduce} \\
Neil Batlivala and Neha Rathi \\
November 16, 2013 \\
\end{center}

\question{I} {Design}
% Clearly explain your design and illustrate its use, being sure to highlight any special features or abilities.

\question{II} {Implementation}
% Describe the portions of the design that are correctly implemented, that have bugs, and that remain unimplemented.

\question{III} {Build, Deploy, and Run}
% Tell us how to cleanly build, deploy, and run your project.

We have provided a config file template.

First, run the master using the following command: 

\begin{center}\texttt{java -jar MapReduceFacility.jar -m <config-file>}\end{center}

Then, programs can be executed from a participant (using the following command:

\begin{center}\texttt{java -jar MapReduceFacility.jar -p <port> -c <cluster-name>}\end{center}

Where \texttt{port} is the MR port that was specified in the config file passed to the master and \texttt{cluster-name} is the cluster name that was specified in the config file passed to the master.

\question{IV} {Dependencies}
% Highlight any dependencies and software or system requirements. 

Our project depends on Java 1.6.0 or higher.

\question{V} {Running and Testing}
% Tell us how to run and test your framework.

Running \texttt{MainServer} will create a registry local to the server's machine and create/bind two \texttt{TestObject}s to this registry. Then, running \texttt{MainClient -c <hostname>} will create a registry local to the client's machine, get the registry local to the server's machine, create/bind two \texttt{TestObject}s to the former, and lookup the two \texttt{TestObjects} from the latter. After this, a series of 17 tests will be invoked on one stub using the two local objects and other stub. As each test passes, a corresponding message will be printed out. Here, we will describe these tests in detail: \\

\textbf{Test 1 (add)}:



\end{document}